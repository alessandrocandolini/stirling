% *****************************************************************************
% 	General commands
% *****************************************************************************

\PassOptionsToPackage{dvipsnames,svgnames}{xcolor}

% un ambiente ad hoc per le citazioni 
% (in teoria gia' coperto dal pacchetto quoting)

% [...] ;-) NO replaced by \textelp{} and other feature of cvsquotes package
%\newcommand{\omissis}{[\dots\negthinspace]}

% Indica a LaTeX la cartella dove sono riposti i file delle immagini
\graphicspath{{./},{./Asymptote/}, {./Images/}, {./Metapost/}}

% Un ambiente ad hoc per gli approfondimenti (definirlo si, ma magari in altro
% modo)
\newenvironment{approfondimento}%
  {\begin{quoting}\small\ignorespaces}%
  {\end{quoting}}

% i.e. and e.g.
% here with thin space (most german convetion), see typography discussion in 
% http://tex.stackexchange.com/questions/25831/canonical-way-to-typeset-spacing-in-abbreviations
\providecommand{\ie}{i.\,e.}
\providecommand{\Ie}{I.\,e.}
\providecommand{\eg}{e.\,g.}
\providecommand{\Eg}{E.\,g.} 

% th
\newcommand{\ordth}{\textsuperscript{th}}

% Cpp 
\def\cpp{\mbox{C++}}

% *****************************************************************************
%       Finer typography	
% *****************************************************************************
%\usepackage[draft,%
%hyphenation,lastparline,nosingleletter,homeoarchy,rivers]{impnattypo}
\usepackage[hyphenation]{impnattypo}
\usepackage[punct-after=false]{fnpct} 
\setfnpct{add-punct-marks=:}
\setfnpct{add-punct-marks=;}
\usepackage[biblatex=true]{embrac}

%\AdaptNoteOpt\footcite\multfootcite
%\AdaptNoteOpt\autocite\multautocite

\AdaptNote\footcite{oo+m}[footnote]{%
\setfnpct{dont-mess-around}%
\IfNoValueTF{#1}
{#NOTE{#3}}
{\IfNoValueTF{#2}{#NOTE[#1]{#3}}{#NOTE[#1][#2]{#3}}}%
}


% *****************************************************************************
% 	Additional hyperref stuff (to be improved)
% *****************************************************************************
\usepackage{hyperref}
\usepackage{xcolor}
\definecolor{RoyalBlue}{cmyk}{1, 0.50, 0, 0}
\definecolor{webbrown}{rgb}{.6,0,0}
\hypersetup{%
    colorlinks=true, linktocpage=true, pdfstartpage=1, pdfstartview=FitV,%
    breaklinks=true, pdfpagemode=UseNone, pageanchor=true, pdfpagemode=UseOutlines,%
    plainpages=false, bookmarksnumbered, bookmarksopen=true, bookmarksopenlevel=1,%
    hypertexnames=true, pdfhighlight=/O,%
    urlcolor=webbrown, linkcolor=RoyalBlue, citecolor=RoyalBlue, pagecolor=RoyalBlue,%
   hyperfootnotes=true,%
% uncomment the following line if you want to have black links (e.g., for printing)
% urlcolor=Black, linkcolor=Black, citecolor=Black, pagecolor=Black,%
    %pdftitle={},%
    %pdfauthor={\textcopyright\ \myName},%
    %pdfsubject={},%
    %pdfkeywords={},%
    %pdfcreator={pdfLaTeX},%
    %pdfproducer={LaTeX con hyperref e ClassicThesis}%
}


\hypersetup{citecolor=webblue}

\newcommand{\mail}[1]{\href{mailto:#1}{\texttt{#1}}}

% *****************************************************************************
% 	Mathematica Packages
% *****************************************************************************

% Save accent for wick.sty (eulervm already downloaded)
\let\oldhat\hat
\let\oldbar\bar

% A.M.S. standard packages.
\usepackage{amssymb}

% Numerical sets (amssymb package required).
\newcommand{\numberset}{\mathbb} 
\providecommand{\N}{\numberset{N}} 
\providecommand{\Z}{\numberset{Z}} 
\providecommand{\Q}{\numberset{Q}} 
\providecommand{\R}{\numberset{R}} 
\providecommand{\C}{\numberset{C}} 

% Dirac notation (braket package required).
\usepackage{braket} 
%\newcommand{\modul}[1]{\mathinner{\vert#1\vert}} 
%\newcommand{\Modul}[1]{\left\vert#1\right\vert} 
%\newcommand{\norm}[1]{\mathinner{\Vert#1\Vert}}
%\newcommand{\Norm}[1]{\left\Vert#1\right\Vert}
\newcommand{\conj}[1]{#1^{*}} 
\newcommand{\adj}[1]{#1^{\dagger}}

% Un ambiente per i sistemi.
\newenvironment{sistema}%
  {\left\lbrace\begin{array}{@{}l@{}}}%
  {\end{array}\right.}
 
% Cool package. (Non male per funzioni speciali e spazitura integrali, ma poco
% customizzabile per integrali multiple)
\usepackage{cool}
\makeatletter
  \Style{ArcTrig=arc}
  \Style{DSymb={{\operator@font d}}}
  \Style{DDisplayFunc=inset,DShorten=true}
  \Style{IntegrateDifferentialDSymb={{\operator@font d}}}
  \makeatother


% Costumizing  
\newcommand{\nto}[3]{#1 = #2, \ldots, #3}
\newcommand{\fullfunction}[3]{#1\colon#2\to#3}

% Intervals 
\usepackage{interval}
\newcommand{\commutator}[2]{\interval[scaled]{#1}{#2}}

% skmath (for differential forms, numeric sets, improvements of functions)
%\usepackage[commonsets]{skmath}
\usepackage{skmath}
\usepackage{skmath_amends}

% *****************************************************************************
% 	Cleveref support for breqn
% *****************************************************************************
\usepackage{cleveref}

% *****************************************************************************
% 	Theorem and Definition Environments.
% *****************************************************************************

\usepackage{amsthm}                       % A.M.S. package for theorems.

\makeatletter
  \newtheoremstyle{classicdef}%           % Stile tipografico dei teoremi
  {11pt}%                                 % Spazio che precede l'enunciato
  {11pt}%                                 % Spazio che segue l'enunciato
  {}%                                     % Stile del font dell'enunciato
  {}%                                     % Rientro (se vuoto, nessun rientro;
  %                                       % \parindent = rientro dei capoversi)
  {\scshape}%                             % Font dell'intestazione
  {:}%                                    % Punteggiatura dopo l'intestazione
  {.5em}%                                 % Spazio che segue l'intestazione:
  %                                       % " " = normale spazio inter-parola;
  %                                       % \newline = a capo
  {}%                                     % Specifica intestazione enunciato
\makeatother

\theoremstyle{classicdef}
%\theoremstyle{definition}
\newtheorem{theorem}{Theorem}[section]
\newtheorem{lemma}{Lemma}[section]
\newtheorem{definition}{Definition}[section]
\newtheorem{exercise}{Exercise}[section]
\newtheorem*{homework}{Homework}
\newtheorem*{hint}{Hint}
\theoremstyle{remark}
\newtheorem*{remark}{Remark}
\renewcommand{\qedsymbol}{\rule{.5em}{.5em}}

% *****************************************************************************
% 	Exercise environment 
% *****************************************************************************

\usepackage{tcolorbox}

% *****************************************************************************
% 	Tensors 
% *****************************************************************************
\usepackage{tensind}
\tensordelimiter{@}

% *****************************************************************************
% 	Linear operators for qm / qft
% *****************************************************************************
\usepackage{linop}

% *****************************************************************************
% 	Breqn
% *****************************************************************************
\usepackage{mathtools}                    % Add support for cramped,
					  % mathlap,etc.

% Add support for Euler if loaded					  
\makeatletter
\@ifpackageloaded{eulervm}
{% if the package was loaded
    \PassOptionsToPackage{euler}{flexisym}
}
{%else:
}
\makeatother 

\usepackage{flexisym}              % flexisym required by breqn
\usepackage{breqn}                        % Breqn
\catcode`\_=8 % for safety
%\catcode`_=12
%\catcode`^=12
					  
\makeatletter
\@ifpackageloaded{eulervm}
{% if the package was loaded
   \def\eqnumsize{\normalfont \Tf@font}   % Add support to Minion Pro
}
{}
\makeatother

\setkeys{breqn}{labelprefix={eq:}}

% *****************************************************************************
% 	Feyn support for breqn
% *****************************************************************************
\usepackage[noglobalbang]{feyn}
\makeatletter
\def\vertexlabel#1#2{
   \setbox0=\hbox to 0pt{\hss$\scriptstyle #2$\hss}
   \begingroup
   \lccode`|=`^
   \lccode`!=`_
  \ifcat #1|
 \endgroup
    \vbox to 0pt{\vss\box0\nointerlineskip\kern2\feyn@maxis}
  \else\ifcat #1!
 \endgroup
    \vbox to 0pt{\kern\feyn@maxis\nointerlineskip\box0\vss}
  \else
 \endgroup
    \PackageError{feyn}
      {Bad arguments for \string\vertexlabel}
      {Usage: \string\vertexlabel^{text} or \string\vertexlabel_{text}}
 \fi\fi
  }
  \makeatother



% *****************************************************************************
% 	Cleveref support for breqn
% *****************************************************************************

\makeatletter
\renewcommand\set@label[2]{\protected@edef\@currentlabel{#2}
\cref@constructprefix{equation}{\cref@result}%
\protected@xdef\cref@currentlabel{%
[equation][\arabic{equation}][\cref@result]\eq@number}
  %%% Work in progress... Support for hyperref 
  \hyper@makecurrent{equation}
  \Hy@raisedlink{\hyper@anchorstart{\@currentHref}}
  \Hy@raisedlink{\hyper@anchorend}
}
\makeatother


% *****************************************************************************
% 	Wick contraction 
% *****************************************************************************

\usepackage{simplewick}% Alternative way to typeset Wick contraction..

% *****************************************************************************
% 	Additional setup and packages
% *****************************************************************************

\hypersetup{citecolor=webbrown}
\hypersetup{pdfstartpage=1}

%\usepackage[perpage, symbol*, stable, multiple]{footmisc}	
%\setlength{\footnotemargin}{.6em}%
%\usepackage{footnpag}
\makeatletter
%      \renewcommand*\thefootnote{\@fnsymbol\c@footnote}%
\renewcommand\@oddfoot{\hfill-- \thepage\ --\hfill}
      \makeatother
\usepackage{footnotebackref}

%\usepackage{nicefrac}

\usepackage{paralist}
\usepackage{cancel}
\usepackage{slashed}

% *****************************************************************************
% 	Asymptote
% *****************************************************************************
\usepackage{asymptote}       % Asymptote Graphical Vector Language.

%\begin{asydef}
%defaultpen(fontsize(10pt));
%texpreamble("\usepackage[opticals,onlytext]{MinionPro}");
%texpreamble("\usepackage[small]{eulervm}");
%texpreamble("\usepackage{nicefrac}");
%texpreamble("\usepackage{cool}");
%\end{asydef}

%\renewcommand{\figurename}{Fig.}
%\renewcommand{\tablename}{TAB.}
\newcommand*\openquote{\makebox(25,-12){\color{lightgray}\scalebox{5}{``}}}
\newcommand*\closequote{\makebox(25,-22){\color{lightgray}\scalebox{5}{''}}}

\Crefname{problem}{L'esercizio}{Gli esercizi}%
\crefname{problem}{l'esercizio}{gli esercizi}%


% Rewrite the appendix macro to avoid side effects on footnotes spacing

%\makeatletter
%\renewcommand*\appendix{\par%
%  \setcounter{chapter}{0}%
%  \setcounter{section}{0}%
%  \gdef\@chapapp{\appendixname}%
%  \gdef\thechapter{\@Alph\c@chapter}%
%  %\csname appendixmore\endcsname
%}
%\makeatother

\newtcolorbox[auto counter,number within=section]{pabox}[2][]{%
   colback=red!5!white,colframe=red!75!black,fonttitle=\bfseries,
   title=Exercise.~\thetcbcounter: #2,#1}

\newtcolorbox[auto counter,number within=section]{paboxx}{%
   colback=red!5!white,colframe=red!75!black,fonttitle=\bfseries,
   title=Remark}


\usepackage{siunitx}


