\documentclass[onecolumn,a4paper,11pt]{article}
\usepackage{jheppub}
\makeatletter
\def\@fpheader{\today}
\makeatother
\pdfoutput=1



% *****************************************************************************
%  Main packages
% *****************************************************************************


\usepackage[english]{babel}       % Attiva supporto per la lingua.
\usepackage[babel]{csquotes}              % Consigliato se si usa biblatex.

\usepackage{graphicx}                     % Per le immagini.

%\usepackage{mparhack}                     % Get marginpar right.
%\usepackage{fixltx2e}                     % Fixes some LaTeX stuff.

%        	\usepackage[opticals,mathlf,onlytext]{MinionPro}%
\usepackage[final]{microtype}             % Microgiustificazione e finezze.

\usepackage{ifpdf}                        % To check if pdflatex is used.

\ifpdf
  \DeclareGraphicsRule{*}{mps}{*}{}       % To include metapost files.
\fi

\usepackage{subfig}                       % Multiple figures.
\usepackage{url}                          % Url.

\usepackage{feynmp}                       % Feynman diagrams via metapost

% *****************************************************************************
% BIBLaTeX
% *****************************************************************************
\usepackage[%
%style=philosophy-modern,%
%style=philosophy-classic,%
style=philosophy-verbose,%
scauthorsbib,%
%refsection=section,%
hyperref,backref,%
backend=biber,%
firstinits=true,%
%square,%
ibidtracker=true]{biblatex}

\addbibresource{bibliography.bib}            % database di biblatex 

% *****************************************************************************
% Quoting 
% *****************************************************************************

\usepackage[font=small, noorphanfirst,
indentfirst=false,
%begintext=\openautoquote,
%endtext=\closeautoquote,
]{quoting}         


% *****************************************************************************
% 	General commands
% *****************************************************************************

\PassOptionsToPackage{dvipsnames,svgnames}{xcolor}

% un ambiente ad hoc per le citazioni 
% (in teoria gia' coperto dal pacchetto quoting)

% [...] ;-) NO replaced by \textelp{} and other feature of cvsquotes package
%\newcommand{\omissis}{[\dots\negthinspace]}

% Indica a LaTeX la cartella dove sono riposti i file delle immagini
\graphicspath{{./},{./Asymptote/}, {./Images/}, {./Metapost/}}

% Un ambiente ad hoc per gli approfondimenti (definirlo si, ma magari in altro
% modo)
\newenvironment{approfondimento}%
  {\begin{quoting}\small\ignorespaces}%
  {\end{quoting}}

% i.e. and e.g.
% here with thin space (most german convetion), see typography discussion in 
% http://tex.stackexchange.com/questions/25831/canonical-way-to-typeset-spacing-in-abbreviations
\providecommand{\ie}{i.\,e.}
\providecommand{\Ie}{I.\,e.}
\providecommand{\eg}{e.\,g.}
\providecommand{\Eg}{E.\,g.} 

% th
\newcommand{\ordth}{\textsuperscript{th}}

% Cpp 
\def\cpp{\mbox{C++}}

% *****************************************************************************
%       Finer typography	
% *****************************************************************************
%\usepackage[draft,%
%hyphenation,lastparline,nosingleletter,homeoarchy,rivers]{impnattypo}
\usepackage[hyphenation]{impnattypo}
\usepackage[punct-after=false]{fnpct} 
\setfnpct{add-punct-marks=:}
\setfnpct{add-punct-marks=;}
\usepackage[biblatex=true]{embrac}

\AdaptNoteOpt\footcite\multfootcite
\AdaptNoteOpt\autocite\multautocite

% *****************************************************************************
% 	Additional hyperref stuff (to be improved)
% *****************************************************************************
\usepackage{hyperref}
\usepackage{xcolor}
\definecolor{RoyalBlue}{cmyk}{1, 0.50, 0, 0}
\definecolor{webbrown}{rgb}{.6,0,0}
\hypersetup{%
    colorlinks=true, linktocpage=true, pdfstartpage=1, pdfstartview=FitV,%
    breaklinks=true, pdfpagemode=UseNone, pageanchor=true, pdfpagemode=UseOutlines,%
    plainpages=false, bookmarksnumbered, bookmarksopen=true, bookmarksopenlevel=1,%
    hypertexnames=true, pdfhighlight=/O,%
    urlcolor=webbrown, linkcolor=RoyalBlue, citecolor=RoyalBlue, pagecolor=RoyalBlue,%
   hyperfootnotes=true,%
% uncomment the following line if you want to have black links (e.g., for printing)
% urlcolor=Black, linkcolor=Black, citecolor=Black, pagecolor=Black,%
    %pdftitle={},%
    %pdfauthor={\textcopyright\ \myName},%
    %pdfsubject={},%
    %pdfkeywords={},%
    %pdfcreator={pdfLaTeX},%
    %pdfproducer={LaTeX con hyperref e ClassicThesis}%
}


\hypersetup{citecolor=webblue}

\newcommand{\mail}[1]{\href{mailto:#1}{\texttt{#1}}}

% *****************************************************************************
% 	Mathematica Packages
% *****************************************************************************

% Save accent for wick.sty (eulervm already downloaded)
\let\oldhat\hat
\let\oldbar\bar

% A.M.S. standard packages.
\usepackage{amssymb}

% Numerical sets (amssymb package required).
\newcommand{\numberset}{\mathbb} 
\providecommand{\N}{\numberset{N}} 
\providecommand{\Z}{\numberset{Z}} 
\providecommand{\Q}{\numberset{Q}} 
\providecommand{\R}{\numberset{R}} 
\providecommand{\C}{\numberset{C}} 

% Dirac notation (braket package required).
\usepackage{braket} 
%\newcommand{\modul}[1]{\mathinner{\vert#1\vert}} 
%\newcommand{\Modul}[1]{\left\vert#1\right\vert} 
%\newcommand{\norm}[1]{\mathinner{\Vert#1\Vert}}
%\newcommand{\Norm}[1]{\left\Vert#1\right\Vert}
\newcommand{\conj}[1]{#1^{*}} 
\newcommand{\adj}[1]{#1^{\dagger}}

% Un ambiente per i sistemi.
\newenvironment{sistema}%
  {\left\lbrace\begin{array}{@{}l@{}}}%
  {\end{array}\right.}
 
% Cool package. (Non male per funzioni speciali e spazitura integrali, ma poco
% customizzabile per integrali multiple)
\usepackage{cool}
\makeatletter
  \Style{ArcTrig=arc}
  \Style{DSymb={{\operator@font d}}}
  \Style{DDisplayFunc=inset,DShorten=true}
  \Style{IntegrateDifferentialDSymb={{\operator@font d}}}
  \makeatother


% Costumizing  
\newcommand{\nto}[3]{#1 = #2, \ldots, #3}
\newcommand{\fullfunction}[3]{#1\colon#2\to#3}

% Intervals 
\usepackage{interval}
\newcommand{\commutator}[2]{\interval[scaled]{#1}{#2}}

% skmath (for differential forms, numeric sets, improvements of functions)
%\usepackage[commonsets]{skmath}
\usepackage{skmath}
\usepackage{skmath_amends}

% *****************************************************************************
% 	Cleveref support for breqn
% *****************************************************************************
\usepackage{cleveref}

% *****************************************************************************
% 	Theorem and Definition Environments.
% *****************************************************************************

\usepackage{amsthm}                       % A.M.S. package for theorems.

\makeatletter
  \newtheoremstyle{classicdef}%           % Stile tipografico dei teoremi
  {11pt}%                                 % Spazio che precede l'enunciato
  {11pt}%                                 % Spazio che segue l'enunciato
  {}%                                     % Stile del font dell'enunciato
  {}%                                     % Rientro (se vuoto, nessun rientro;
  %                                       % \parindent = rientro dei capoversi)
  {\scshape}%                             % Font dell'intestazione
  {:}%                                    % Punteggiatura dopo l'intestazione
  {.5em}%                                 % Spazio che segue l'intestazione:
  %                                       % " " = normale spazio inter-parola;
  %                                       % \newline = a capo
  {}%                                     % Specifica intestazione enunciato
\makeatother

\theoremstyle{classicdef}
%\theoremstyle{definition}
\newtheorem{theorem}{Theorem}[section]
\newtheorem{lemma}{Lemma}[section]
\newtheorem{definition}{Definition}[section]
\newtheorem{exercise}{Exercise}[section]
\newtheorem*{homework}{Homework}
\newtheorem*{hint}{Hint}
\theoremstyle{remark}
\newtheorem*{remark}{Remark}
\renewcommand{\qedsymbol}{\rule{.5em}{.5em}}

% *****************************************************************************
% 	Exercise environment 
% *****************************************************************************

\usepackage{tcolorbox}

% *****************************************************************************
% 	Tensors 
% *****************************************************************************
\usepackage{tensind}
\tensordelimiter{@}

% *****************************************************************************
% 	Linear operators for qm / qft
% *****************************************************************************
\usepackage{linop}

% *****************************************************************************
% 	Breqn
% *****************************************************************************
\usepackage{mathtools}                    % Add support for cramped,
					  % mathlap,etc.

% Add support for Euler if loaded					  
\makeatletter
\@ifpackageloaded{eulervm}
{% if the package was loaded
    \PassOptionsToPackage{euler}{flexisym}
}
{%else:
}
\makeatother 

\usepackage{flexisym}              % flexisym required by breqn
\usepackage{breqn}                        % Breqn
\catcode`\_=8 % for safety
%\catcode`_=12
%\catcode`^=12
					  
\makeatletter
\@ifpackageloaded{eulervm}
{% if the package was loaded
   \def\eqnumsize{\normalfont \Tf@font}   % Add support to Minion Pro
}
{}
\makeatother

\setkeys{breqn}{labelprefix={eq:}}

% *****************************************************************************
% 	Feyn support for breqn
% *****************************************************************************
\usepackage[noglobalbang]{feyn}
\makeatletter
\def\vertexlabel#1#2{
   \setbox0=\hbox to 0pt{\hss$\scriptstyle #2$\hss}
   \begingroup
   \lccode`|=`^
   \lccode`!=`_
  \ifcat #1|
 \endgroup
    \vbox to 0pt{\vss\box0\nointerlineskip\kern2\feyn@maxis}
  \else\ifcat #1!
 \endgroup
    \vbox to 0pt{\kern\feyn@maxis\nointerlineskip\box0\vss}
  \else
 \endgroup
    \PackageError{feyn}
      {Bad arguments for \string\vertexlabel}
      {Usage: \string\vertexlabel^{text} or \string\vertexlabel_{text}}
 \fi\fi
  }
  \makeatother



% *****************************************************************************
% 	Cleveref support for breqn
% *****************************************************************************

\makeatletter
\renewcommand\set@label[2]{\protected@edef\@currentlabel{#2}
\cref@constructprefix{equation}{\cref@result}%
\protected@xdef\cref@currentlabel{%
[equation][\arabic{equation}][\cref@result]\eq@number}
  %%% Work in progress... Support for hyperref 
  \hyper@makecurrent{equation}
  \Hy@raisedlink{\hyper@anchorstart{\@currentHref}}
  \Hy@raisedlink{\hyper@anchorend}
}
\makeatother


% *****************************************************************************
% 	Wick contraction 
% *****************************************************************************

\usepackage{wick}% Very powerful, but some troubles with AMSmath accents
\usepackage{simplewick}% Alternative way to typeset Wick contraction..

% Naive way to save accents inside wick environment (the changes do not effect
% the behavior of accents outside wick macro)

\def\newhat#1{{\oldhat{#1}}}
\def\newbar#1{{\oldbar{#1}}}

%\def\WKsep{.5}    

\let\oldwick\wick
\renewcommand{\wick}[3][o]
{\begingroup
\let\hat\newhat
\let\bar\newbar
\oldwick[#1]{#2}{#3}
\endgroup
}

\let\oldWwick\Wwick
\renewcommand{\Wwick}[4]
{\begingroup
\let\hat\newhat
\let\bar\newbar
\oldWwick{#1}{#2}{#3}{#4}
\endgroup
}

% *****************************************************************************
% 	Additional setup and packages
% *****************************************************************************

\hypersetup{citecolor=webbrown}
\hypersetup{pdfstartpage=1}

%\usepackage[perpage, symbol*, stable, multiple]{footmisc}	
%\setlength{\footnotemargin}{.6em}%
%\usepackage{footnpag}
\makeatletter
%      \renewcommand*\thefootnote{\@fnsymbol\c@footnote}%
\renewcommand\@oddfoot{\hfill-- \thepage\ --\hfill}
      \makeatother
\usepackage{footnotebackref}

%\usepackage{nicefrac}

\usepackage{paralist}
\usepackage{cancel}
\usepackage{slashed}

% *****************************************************************************
% 	Asymptote
% *****************************************************************************
\usepackage{asymptote}       % Asymptote Graphical Vector Language.

%\begin{asydef}
%defaultpen(fontsize(10pt));
%texpreamble("\usepackage[opticals,onlytext]{MinionPro}");
%texpreamble("\usepackage[small]{eulervm}");
%texpreamble("\usepackage{nicefrac}");
%texpreamble("\usepackage{cool}");
%\end{asydef}

%\renewcommand{\figurename}{Fig.}
%\renewcommand{\tablename}{TAB.}
\newcommand*\openquote{\makebox(25,-12){\color{lightgray}\scalebox{5}{``}}}
\newcommand*\closequote{\makebox(25,-22){\color{lightgray}\scalebox{5}{''}}}

\Crefname{problem}{L'esercizio}{Gli esercizi}%
\crefname{problem}{l'esercizio}{gli esercizi}%


% Rewrite the appendix macro to avoid side effects on footnotes spacing

%\makeatletter
%\renewcommand*\appendix{\par%
%  \setcounter{chapter}{0}%
%  \setcounter{section}{0}%
%  \gdef\@chapapp{\appendixname}%
%  \gdef\thechapter{\@Alph\c@chapter}%
%  %\csname appendixmore\endcsname
%}
%\makeatother

\newtcolorbox[auto counter,number within=section]{pabox}[2][]{%
   colback=red!5!white,colframe=red!75!black,fonttitle=\bfseries,
   title=Exercise.~\thetcbcounter: #2,#1}

\newtcolorbox[auto counter,number within=section]{paboxx}{%
   colback=red!5!white,colframe=red!75!black,fonttitle=\bfseries,
   title=Remark}


\usepackage{siunitx}


          % Un file personale:
					  % permette di mantenere ordinato il
					  % sorgente

					  \newcommand{\asympt}{\sim}
%					  \newcommand{\BigO}[1]{\ensuremath{\operatorname{O}\bigl(#1\bigr)}}
					  \newcommand{\bigO}{\ensuremath{\operatorname{O}}}
\title{Handout  on Stirling's formula}
\author{Alessandro Candolini}
\affiliation{Department of Physics, University Of Trieste, via Valerio~2,
Trieste, Italy}
\emailAdd{alessandro.candolini@gmail.com}
\date{\today}
\abstract{
Some proofs of the well-known Stirling's asymptotic approximation of the
factorial (or the Gamma function) are sketched. No attempt was made at full mathematical rigour, 
emphasis is mostly on the main ideas behind these proofs. }
\begin{document}
\maketitle

%    \tableofcontents

\section{Overview}\label{sec:overview}
The ``factorial'' $\Factorial{n}$ of a positive integer number $n\in\N$ is defined
by induction:
let $\Factorial{0} = 1$ and $\Factorial{n} = n \Factorial{(n-1)}$ for every $n\geq 1$.

Stirling's formula
is an asymptotic approximation of $\Factorial{n}$ for ``large'' $n$, namely
\begin{dmath}[label={stirling}]
   \Factorial{n} \asympt  \sqrt{2\pi} n^{n+ \frac{1}{2}} \exp{-n} 
   \condition{as
      $n \rightarrow+\infty$}
\end{dmath},
where the symbol $\asympt$ will be used to denote asymptotic equality.
The rigorous meaning of 
\cref{eq:stirling} is
   \begin{dmath*}[compact]
   \lim_{n\rightarrow +\infty} \frac{\Factorial{n}}{\sqrt{2\pi}
   n^{n+\frac{1}{2}} \exp{-n}} = 1 
\end{dmath*}.

Since factorials are among the basic ingredients of combinatorics, Stirling's
formula is extremely useful in many fields of mathematics, statistics and
physics to help simplify formul\ae{} involving factorials. For this reason, it
seems useful to review few of the standard methods to prove it.

The simplest argument leads to an (heuristic) integral approximation of
$\log*{\Factorial{n}}$ (see \cref{sec:int bounds}). To be precise, the method
actually returns rigorous lower and upper integral bounds of
$\log*{\Factorial{n}}$. A (still heuristic)  refinement of  the procedure allows
to  extract the right dependence on $n$, but the procedure is still heuristic
and the factor $\sqrt{2\pi}$ is still missing.  In order to fully recover
\cref{eq:stirling} in a rigorous way, more advanced machinary is needed.

The heuristic result of \cref{sec:int bounds} is used as a guess for the
rigorous treatment of \cref{sec:Wallis}, based on the famous Wallis formula.
Wallis formula is a rather technical one however, and it would be useful if it
were possible to prove Stirling's formula relying on more general tools.

Remember that factorials can be written in terms of the gamma trascendental
function:
\begin{dmath}[label={gamma n}, compact]
   \Factorial{n} = \GammaFunc{n+1} = \Integrate{t^{n} \exp{-t}}{t, 0, +\infty} 
\condition*{\forall n \in \N}
\end{dmath},
where the Euler's integral representation of the Gamma function has been used in
\cref{eq:gamma n}.  Gamma function provides a generalization of the factorial to
real and complex numbers. This generalization is unique up to some extra
conditions.  We will thoroughly study the basic properties of Gamma function and
how it generalizes factorials in \cref{sec:Gamma}.  The Euler's integral
representation of the Gamma function in \cref{eq:gamma n} is a good starting
point to prove \cref{eq:stirling} using more advanced techniques.

A standard approach is that of using the saddle-point asymptotic method on the
right-hand side of \cref{eq:gamma n} , which leads directly to
\cref{eq:stirling}.  This approach is outlined in \cref{sec:spm}.  Saddle point
method is a powerful method to compute the asymptotic behavior of certain
parametric integrals, and it  is discussed more deeply in \cref{app:saddle}.

A third way to prove \cref{eq:stirling} is given in \cref{sec:Michael}. Only
basic  tools of calculus courses are needed. This is a recent proof due to
R.~Michael, which again involves the integral representation \cref{eq:gamma n}.

\section{Naive integral approximation\label{sec:int bounds}}

By definition, the factorial of $n\in\N$ is a product.  For every
$n\in\N\backslash\{0\}$,  $\Factorial{n}$ can be equivalently written as
$\Factorial{n} = \prod_{k=1}^{n} k$%
\footnote{%
   For every $n\in\N\backslash\{0\}$, this is equivalent to the definition of
   the factorial by induction given at the beginning of \cref{sec:overview}
   (proof of the equivalence is by induction). For $n=0$ instead, the formula
   $\Factorial{n} = \prod_{k=1}^{n} k$  does not work.  
   The position $\Factorial{0}=1$ is conventional and
   arbitrary, however it is useful and it is consistent with the relation
   between factorials and the gamma function (see \cref{sec:Gamma}).
}.
Now, the trick to proceed is this: 
\begin{inparaenum}[(a)]
\item first of all, turn this product into a sum by taking the logarithm of it,
   then
\item approximate this sum with an integral.  
\end{inparaenum} 
Let us explain the various steps in detail.

\begin{figure}
\centering
\asyinclude[inline=true]{./Asymptote/Integral.asy}
\caption{%
   Illustration of the key idea beyond integral method: comparison between sums
   and integrals.  Since $\log{x} $ is an increasing function of $x$, the
   integral of $\log{x}$ (graphical meaning: gray area in figure) understimates
   $\log*{\Factorial{n}}$ (the total area under the rectangles), the (nearly
   triangular) brown areas above the curve of $\log{x}$ being left out. On the
   contrary, the integral of $\log{x+1}$ overstimates that sum.  
   \label{fig:int met}
}
\end{figure}

First, for $n\in\N\backslash\{0\}$ take the logarithm of $\Factorial{n}$:
\begin{dmath}[label={logn!},compact]
   \log*{\Factorial{n}} = \log{\prod_{k=1}^{n} k }= \sum_{k=1}^{n}
   \log{k} \condition*{n\in\N\backslash\{0\}}.
\end{dmath}
\Cref{eq:logn!}  is an \emph{exact} expression of the logarithm of the factorial.  Now, \emph{approximate} the sum in \cref{eq:logn!} with an integral:
\begin{dmath}[label={int approx},compact]
   \log*{\Factorial{n}} = \sum_{k=1}^{n} \log k \approx \Integrate{\log x}{x,1,n}
\condition*{n\in\N\backslash\{0\}}
\end{dmath}.
Graphical interpretation of \cref{eq:logn!} (see \cref{fig:int met}):
$\sum_{k=1}^{n} \log k$ (area under the rectangles) is approximated by
$\Integrate{\log x}{x,1,n}$ (whose pictorial meaning is the gray area below the
curve of $\log x $).

The integral in \cref{eq:int approx} is evaluated 
by parts as usual,  
\begin{dmath*}[compact]
\Integrate{\log x }{x, 1, n}
= \Big[ x\log x -  x\Big]_{x=1}^{x=n} =  n \log n -
n+1
\end{dmath*},
hence \cref{eq:int approx}  reads
\begin{dmath}[label={int heuristic}, compact]
   \log{\Factorial{n}} \approx n \log n - n +1
\end{dmath}.
For ``large'' $n$,  $n-1\approx n$ and 
exponentiating \cref{eq:int heuristic} yields
\begin{dmath}[frame,compact]
   \Factorial{n} \approx \exp{n\log n - n } = n^{n} \exp{-n}  
\end{dmath}.
At this point however we have no precise control of the kind of approximation
(\ie, a bound on the error) in
\cref{eq:int approx}.

Actually, the approximation  \cref{eq:int approx} \emph{under}stimates $\log
{\Factorial{n}}$, as it 
should be 
clear from \cref{fig:int met}: the area under the
curve of $\log x$ 
is less than the area under the rectangles, the (brown) areas above
the curve of $\log x$  being left out. 

\begin{approfondimento}
   If you don't want to rely on \cref{fig:int met}, it is not difficult to
   convert the graphical argument into a purely algebraic proof.
Since $\log x$ is a \emph{monotonically 
   increasing} function of $x$, then 
\begin{dmath*}[compact]
   \Integrate{\log x}{x,k-1,k} < \log k < \Integrate{\log x}{x, k, k+1}
%   \condition{for every $k\in\N$, $k\geq 2$}
\end{dmath*},
for every $k\in\N$, $k\geq 2$.
Equivalently, by changing variables in the second integral:
\begin{dmath*}[compact]
   \Integrate{\log x}{x,k-1,k} < \log k < \Integrate{\log{x+1}}{x, k-1, k}
\end{dmath*},
for every $k\in\N$, $k\geq 2$.
Summing from $k=2$ to $k=n$ yields
\begin{dmath*}[compact]
\sum_{k=2}^{n} 
   \Integrate{\log x}{x,k-1,k}  <
\underbrace{%
\sum_{k=2}^{n}\log k
}_{\mathclap{\left( \sum_{k=1}^{n} \log k\right) - \log 1 %
      = \sum_{k=1}^{n} \log k = \log {\Factorial{n}}  }  }
<
\sum_{k=2}^{n} 
\Integrate{\log{x+1}}{x, k-1, k}
\end{dmath*}, 
and therefore
\begin{dmath*}[compact]
   \Integrate{\log x }{x,1,n} < \log {\Factorial{n}} <
   \Integrate{\log{x+1}}{x,1,n}
\end{dmath*},
for every $n\in\N$, $n\geq 2$,
or if you prefer to include also the case $n=1$
\begin{dmath*}[compact]
   \Integrate{\log x }{x,1,n} \leq \log{\Factorial{n}}\leq 
   \Integrate{\log{x+1}}{x,1,n}
   \condition{for every $n\in\N \backslash\{0\}$}
\end{dmath*}.
Graphical meaning (refer to \cref{fig:int met}):
the total area of the rectangles
(\ie, $\log{\Factorial{n}}$) is greater than $\Integrate{\log x}{x,1,n}$
(\ie, the area below $\log x$)  and less than
$\Integrate{\log{x+1}}{x,1,n}$ [\ie, the area below $\log {x+1}$].
This shows that the approximation \cref{eq:int approx} actually gives a
rigorous \emph{lower} bound of $\log{\Factorial{n}}$.
\end{approfondimento}



The approximation can be 
improved by including the
contributions coming from the nearly-triangular brown areas left-out above the
plot of $\log x$ (see \cref{fig:int met}).
Notice that this is nothing but the usual trapezoid rule to approximate an
integral.%
\footnote{Euler-MacLaurin formula gives the error of the trapezoid  approximation.
 We will come back to this point discussing Stirling asymptotic series.}

The $k$-th left out area can be roughly approximated by the triangular area 
\begin{dmath*}
\frac{1}{2} \left[ \log {k+1} - \log k \right ] 
\end{dmath*}.
Summing all the contributions yields
%\begin{equation}\label{eq:int+tsum}
   \begin{dmath}[label={int+tsum}]
      \log{\Factorial{n}}\approx \Integrate{ \log x }{t,1, n} + \frac{1}{2} \sum_{k=1}^{n}
\left[ \log  {k+1} - \log k \right] 
\end{dmath}.
The telescopic sum on the right-hand side can be evaluated exactly
\begin{dmath}
\sum_{k=1}^{n} \left[ \log {k+1} - \log k \right] = \log n - \log 1 = \log n
\end{dmath},
(the rigorous proof is by induction) and
\begin{dmath}[compact]
   \log{\Factorial{n}} \approx n \log n - n+1 + \frac{1}{2}\log n 
\approx  \left( n + \frac{1}{2} \right) \log n -n  
\end{dmath},
and by taking the exponential 
\begin{dmath}[label={int guess exp}]
   \Factorial{n} \approx n^{n+\frac{1}{2}} \exp{-n} 
\end{dmath}
This of course is just an heuristic argument, since at this stage we have no control 
on the
kind of approximation occurring in \cref{eq:int+tsum}.
\Cref{eq:int guess exp} shares the correct dependence on $n$, compared to
\cref{eq:stirling}, but the factor
$\sqrt{2\pi}$ is still missing. 
In the next section we will use this result as  starting point for a
more rigorous treatment. 


\section{Rigorous results from integral method. Wallis formula\label{sec:Wallis}}
Given the heuristic result \cref{eq:int guess exp},
define a sequence $(a_{n})_{n\in\N\backslash\{0\}}$ of positive numbers
\begin{dmath*}
   a_{n} = \frac{\Factorial{n}}{n^{n+\frac{1}{2}} \exp{-n}}  
\condition*{\forall n\in\N}
\end{dmath*}.
Our goals are:
\begin{inparaenum}[(a)]
\item prove that $\lim_{n} a_{n}$ exists;
\item calculate it.
\end{inparaenum}
This will be done in \cref{thm:stirling1}.
To achieve this,
we need the following preliminary result.
\begin{theorem}[Wallis formula]
   \label{thm:Wallis}
\begin{dmath}[label={Wallis}]
   \lim_{n\rightarrow+\infty} \frac{2^{2n} (\Factorial{n})^{2}}{\sqrt{n}
      \Factorial{(2n)}} = \sqrt{\pi}
\end{dmath}.
\end{theorem}
\begin{proof}
A elegant proof based on properties of gamma function is given in
\cref{sec:Gamma}. Here, we give another proof, based only on elementary
calculus. The drawback of this proof is that 
it becomes quite long and  a bit tricky.

Consider the sequence of integrals
\begin{dmath}[label={In}]
I_{n} = 
\Integrate{\sin[n]{\vartheta}}{\vartheta, 0, \frac{\pi}{2}} 
\condition*{\forall n\in\N}
\end{dmath}.
Integrals of this form often arise in physics, \eg, algebraic relations of
non-relativistic quantum
theory of angular momentum, \ie, spherical harmonics%
\autocite[See for example][\S~3.6]{Sakurai.Napolitano:2011}.
We will use those integrals to prove the Wallis formula. 
First of all, we need an explicit expression for $I_{n}$.
These integrals can be easily computed by using the beta trascendental function, see
\cref{sec:Gamma}. Here we will use a naive approach: we will use integration by
parts to obtain a linear recursive equation for $I_{n}$, which can be solved
exactly in closed form.  Integration by parts yields
\begin{dmath*}
I_{n} = 
\Integrate{\sin[n]{\vartheta}}{\vartheta, 0, \frac{\pi}{2}} 
= - \left. \cos{\vartheta} \sin[n-1]{\vartheta} \right|_{0}^{\frac{\pi}{2}} +
(n-1) \Integrate{ \cos[2]{\vartheta} \sin[n-2]{\vartheta} }{\vartheta, 0, \frac{\pi}{2}}
= (n-1)  \left( I_{n-2} - I_{n} \right) 
\end{dmath*}.
Re-arranging the terms in this expression yields
\begin{dmath}[label={recursive}]
   I_{n} = \frac{n-1}{n} I_{n-2}  \condition{$\forall n\in\N$ and $n\geq 2$}
\end{dmath}.
\Cref{eq:recursive} is a second-order linear homogeneus recursive equation with
non-constant (rational)
coefficients. 
The initial 
conditions are
\begin{dgroup*}
   \begin{dmath*}[compact]
      I_{0} = \Integrate{}{\vartheta, 0, \frac{\pi}{2}} = \frac{\pi}{2} 
   \end{dmath*},
   \begin{dmath*}[compact]
I_{1} = \Integrate{\sin \vartheta}{\vartheta, 0, \frac{\pi}{2}} =  \left.
- \cos \vartheta \right|_{0}^{\frac{\pi}{2}} = 1 
\end{dmath*}.
\end{dgroup*}
One can use the general theory of linear recursive equations with variable
(polynomial) coefficients to approach
\cref{eq:recursive} systematically. The fact that the coefficients are not
constant of course makes the things harder. The general solution of
\cref{eq:recursive} can be written in terms of gamma function. 
At a more straightforward level, one can start 
by calculating explicitly the lowest terms in the sequence $I_{n}$ by using the
initial conditions,
sequentially producing the next tems until a clear pattern emerges.
Once the general patter is identified, one can prove 
rigorously by induction that
the pattern is indeed correct. Let us proceed in this way.

For $n$ odd, $n=2k+1$ for some $k\in\N$, the first terms in the recursion are
\begin{dgroup*}
   \begin{dmath*}
I_{3} = \frac{2}{3} I_{1} 
\end{dmath*},
\begin{dmath*}[compact]
I_{5} = \frac{4}{5} I_{3} = \frac{4}{5} \frac{2}{3} I_{1} 
\end{dmath*},
\begin{dmath*}[compact]
I_{7} = \frac{6}{7} I_{5} = \frac{6}{7} \frac{4}{5} \frac{2}{3} I_{1} 
\end{dmath*},
\end{dgroup*}
and it is possible to identify a pattern:
\begin{dmath}[compact, frame, label={I2k+1}]
   I_{2k+1} = \frac{\DblFactorial{(2k)}}{\DblFactorial{(2k+1)}} I_{1} =
   \frac{\DblFactorial{(2k)}}{\DblFactorial{(2k+1)}}
   \condition*{\forall k \in \N}
\end{dmath}.
(Double factorial notation is explained in \cref{sec:Gamma}.)
The rigorous proof of the relation above is by induction.

For $n$ even, $n = 2k$ for some $k\in\N$, the first terms in the recursion are
\begin{dgroup*}
\begin{dmath*}[compact]
I_{2} = \frac{1}{2} I_{0} 
\end{dmath*},
\begin{dmath*}[compact]
I_{4} = \frac{3}{4} I_{2} = \frac{3}{4} \frac{1}{2} I_{0} 
\end{dmath*},
\begin{dmath*}[compact]
I_{6} = \frac{5}{6} I_{4} = \frac{5}{6} \frac{3}{4} \frac{1}{2} I_{0} 
\end{dmath*},
\end{dgroup*}
and we may expect the general pattern to be 
\begin{dmath}[compact,frame,label={I2k}]
   I_{2k} = \frac{\DblFactorial{(2k-1)}} {\DblFactorial{(2k)}} I_{0} =
   \frac{\DblFactorial{(2k-1)}}{\DblFactorial{(2k)}} \frac{\pi}{2}
\end{dmath}.
Also in this case, the rigorous proof can be done by induction.

Now that we have explicit formul\ae{} for $I_{n}$ in \cref{eq:In}, we can ask:
how are those integrals related to the Wallis formula \cref{eq:Wallis}? How can $I_{n}$ be useful
to prove \cref{eq:Wallis}? This is a rather technical point: the trick is using
$I_{n}$ to obtain a chain of inequalities which will allow us to prove \cref{eq:Wallis} by
invoking the \emph{squeeze theorem}.

First of all, since $0 \leq \sin\vartheta \leq 1 $ for $\vartheta \in
\interval{0}{\pi/2}$,  we have
\begin{dmath*}[compact]
   0 \leq \sin[2k+1]{\vartheta} \leq \sin[2k]{\vartheta} \leq
   \sin[2k-1]{\vartheta}
   \condition{$\forall \vartheta\in \interval{0}{\pi/2}$ and $\forall k \in
      \N\backslash\{0\}$}
\end{dmath*}
and for the monotonicity of the integral we have
\begin{dmath*}[compact]
0 \leq I_{2k+1} \leq I_{2k} \leq I_{2k-1}  
\condition*{\forall k \in \N\backslash\{0\}}
\end{dmath*}.
Thus,
\begin{dmath*}[compact]
0 \leq \frac{(2k)!!}{(2k+1)!!} \leq \frac{(2k-1)!!}{(2k)!!} \frac{\pi}{2} \leq
\frac{(2k-2)!!}{(2k-1)!!}  
\condition*{\forall k \in \N\backslash\{0\}}
\end{dmath*},
or 
\begin{dmath*}[compact]
0 \leq \frac{(2k)!!}{(2k+1)!!} \leq \frac{(2k+1)!!}{(2k)!!} \frac{1}{2k+1} \frac{\pi}{2} \leq
\frac{(2k)!!}{(2k+1)!!}  \frac{2k+1}{2k} 
\condition*{\forall k \in \N\backslash\{0\}}
\end{dmath*}.
%for every $k\in\N\backslash\{0\}$.
Since
\begin{dmath*}[compact]
   \frac{\DblFactorial{(2k)}}{\DblFactorial{(2k+1)}} = \frac{\left(
	 \DblFactorial{(2k)}\right)^{2}}{\Factorial{(2k+1)}} = 
   \frac{
      \left( 2^{k} \Factorial{k} \right)^{2}}{\Factorial{(2k+1)}} = \frac{2^{2k}
      (\Factorial{k})^{2}}{\Factorial{(2k)}}
\frac{1}{2k+1} 
\end{dmath*},
we get
\begin{dmath*}[compact]
   0 \leq \frac{1}{2k+1} \frac{2^{2k} (\Factorial{k})^{2}}{\Factorial{(2k)}}
   \leq \frac{\Factorial{(2k)}}{2^{2k}
      (\Factorial{k})^{2}} \frac{\pi}{2} \leq \frac{2^{2k}
      (\Factorial{k})^{2}}{\Factorial{(2k)}} \frac{1}{2k} 
\end{dmath*}.
Moving things around,
\begin{dmath*}[compact]
   0 \leq \frac{k}{2k+1} \left( \frac{2^{2k} (\Factorial{k})^{2}}{\sqrt{k}
	 \Factorial{(2k)}} \right)^{2}
   \leq  \frac{\pi}{2} \leq \frac{1}{2}  \left( \frac{2^{2k} (\Factorial{k})^{2}}{\sqrt{k}
	 \Factorial{(2k)}} \right)^{2} 
\end{dmath*}.
It follows that
\begin{dgroup*}
   \begin{dmath*}
      \left( \frac{2^{2k} (\Factorial{k})^{2}}{\sqrt{k} \Factorial{(2k)}}
      \right)^{2} \leq \frac{2k+1}{k} \frac{\pi}{2}
   \end{dmath*}
   \begin{dsuspend}
   and
\end{dsuspend}
   \begin{dmath*}
      \left( \frac{2^{2k} (\Factorial{k})^{2}}{\sqrt{k} \Factorial{(2k)}} \right)^{2} \geq \pi   
   \end{dmath*}
\end{dgroup*}
Since $\lim_{k\rightarrow+\infty} \frac{2k+1}{k} \frac{\pi}{2}= \pi $, 
applying squeeze theorem yields
\begin{dmath*}
   \lim_{k\rightarrow+\infty} \left( \frac{2^{2k} (\Factorial{k})^{2} }{\sqrt{k}
	 \Factorial{(2k)}}
\right)^{2} = \pi 
\end{dmath*}.
Taking the square of both sides yields \cref{eq:Wallis}.
\end{proof}
%\end{document}

We are ready to prove Stirling formula.

\begin{theorem}[Stirling formula]
   \label{thm:stirling1}
   Consider the sequence $(a_{n})_{n\in\N\backslash\{0\}}$ definied by
   \begin{dmath}[label={an}]
    a_{n} = \frac{\Factorial{n}}{n^{n+\frac{1}{2}} \exp{-n}}  
\condition*{\forall n\in\N}
\end{dmath}.
Then, 
\begin{dmath}[label={liman}]
   \lim_{n\rightarrow +\infty} a_{n} = \sqrt{2\pi}
\end{dmath}.
\end{theorem}.

\begin{proof}
The proof has two steps: 
\begin{itemize}
\item 
   show existence of the limit in \cref{eq:liman}, \ie, prove that there exists a real number $c\in\R$ such that
\begin{dmath*}
\lim_{n} \frac{ n!}{n^{n+\frac{1}{2}}\exp{-n}} = c  
\end{dmath*},
\item
   then apply Wallis formula (\cref{thm:Wallis}) to show that $c=\sqrt{2\pi}$. 
\end{itemize}

To prove the first part, we will show now that the sequence $(\alpha_{n})_{n}$ is decreasing and
bounded below by a positive constant, thus it converges to some positive real
number $c$.

Consider  
\begin{dmath*}
   \log{\frac{a_{n}}{a_{n+1}}} =  \left( n + \frac{1}{2} \right)  \log{1 +
      \frac{1}{n}} -1 
\end{dmath*}.
Remember the Taylor series of log:
\begin{dmath*}
   \log{1+x} = \sum_{k=1}^{+\infty} (-1)^{k-1} \frac{x^{k}}{k} 
\end{dmath*},
for $\abs{x}<1$. The first terms are
\begin{dmath*}
   \log{1+x}= x - \frac{x^{2}}{2} + \frac{x^{3}}{3} -\frac{x^4}{4}+ \bigO(x^{5}) 
\end{dmath*}
Thus
\begin{dmath*}
   \log{1 + \frac{1}{n}}= \frac{1}{n} - \frac{1}{2n^{2}} +
\frac{1}{3n^{3}} -\frac{1}{4n^4} \bigO\left( n^{-5} \right) 
\end{dmath*},
and
\begin{dmath*}
   \log{\frac{a_{n}}{a_{n+1}}} = 
   \left( n + \frac{1}{2} \right) \left( \frac{1}{n} -
\frac{1}{2n^{2}} + \frac{1}{3n^{3}} -\frac{1}{4n^{4}} + \bigO \left( n^{-5} \right) \right) -1 
= 1 - \frac{1}{2n} + \frac{1}{3n^{2}} + -\frac{1}{4n^3} + \frac{1}{2n} -
\frac{1}{4n^{2}} +\frac{1}{6n^3} -1 +
\bigO\left( n^{-4}\right)
= \frac{1}{3n^{2}} - \frac{1}{4n^{2}} - \frac{1}{4n^3} + \frac{1}{6n^3} +\bigO
\left( n^{-4} \right) 
= \frac{1}{12 n^{2}} -\frac{1}{12n^3} + \bigO \left( n^{-4} \right)  
\end{dmath*}
This implies $\log{ a_{n} / a_{n+}}>0$ and thus $(a_{n})_{n}$ is a
\emph{descresing} sequence (at least for
sufficiently large $n$). 

To prove that $c=\sqrt{2\pi}$, consider the following limit:
\begin{dmath*}
\lim_{n} 
\underbrace{%
\frac{2^{2n} (n!)^{2}}{\sqrt{n} (2n)!} 
}_{\rightarrow \sqrt{\pi}}\underbrace{%
\left(
\frac{n^{n+\frac{1}{2}}\exp{-n}}{n!} \right)^{2} 
}_{\rightarrow \frac{1}{c^{2}}}\underbrace{%
\frac{(2n)!}{(2n)^{2n+\frac{1}{2}}
\exp{-2n}} 
}_{\rightarrow c } = \frac{\sqrt{\pi}}{c} 
\end{dmath*}.
(Here we have used the theorem of the limit of a product.)
A direct computation shows that this limit is also equal to
\begin{dmath*}
\frac{\sqrt{\pi}}{c} =
\lim_{n\rightarrow+\infty} \frac{2^{2n}  n^{2 \left( n +\frac{1}{2}
\right)} \exp{-2n}}{\sqrt{n}  (2n)^{2n+\frac{1}{2} } \exp{-2n}
} 
= \lim_{n\rightarrow+\infty} \frac{2^{2n} n^{2n+1}}{n^{\frac{1}{2}}
(2n)^{2n +\frac{1}{2}}} 
= \lim_{n\rightarrow+\infty} 2^{2n} \left( \frac{n}{2n} \right)^{2n +
\frac{1}{2}} 
= \frac{1}{\sqrt{2}}
\end{dmath*},
from which it follows that $ c= \sqrt{2\pi}$.
\end{proof}


\section{Saddle point derivation of Stirling's formula\label{sec:spm}}

This is a standard way (and probably the simplest one) for proving Stirling's formula.
The method however requires some advanced material:
\begin{itemize}
   \item Relation between factorial and gamma function;
   \item properties of the gamma function (in particular, its integral
      representation);
   \item saddle point method.
\end{itemize}
The relation between factorial and gamma function and properties of gamma
function are highlighted in
\cref{sec:Gamma}.
The saddle point method is summarized in \cref{sec:saddle}.
In this section we
will discuss how this works for the factorial of a natural number (which is rather simple).
The full application of the method to discuss the asymptotic behavior of the
gamma function in the complex plane is done in the appendix. 

The foundamental relation between factorials and the Euler's gamma
function is
\begin{dmath*}[compact]
   \Factorial{n} = \GammaFunc{n+1} = \Integrate{ t^{n} \exp{-t}}{t, 0 , +\infty} 
   \condition*{\forall n \in \N}
\end{dmath*},
where we have used the integral representation of the gamma function.
(See \cref{sec:Gamma} for details.)

Preliminary step: let us write the integrand in a completely equivalent
fashion:
\begin{dmath*}[compact]
   \GammaFunc{n+1} =  
   \Integrate{\exp{\log{t^{n}}} \exp{-t}}{t, 0, +\infty} =  
   \Integrate{\exp{n \log{t} - t }}{t, 0, +\infty} 
\end{dmath*}.

\begin{remark}
The latter was not a change of variables within the integral. We have just
used the fact $t^{n}  = \cramped{\exp{\log{t^{n}}}} = \exp{n \log{t}}$.
\end{remark}


In order to understand better what is going on, let us briefly outline the
behavior of the integrand, which looks as shown in 
\cref{fig:texp}.
\begin{figure}
\centering
\asyinclude[inline=true]{./Asymptote/texp.asy}
\caption{Plot of the integrand function $g(t) = t^{n} \exp{-t}$ and its}
\label{sec:Gamma}
\end{figure}

Let $\fullfunction{f,g}{\interval[open right]{0}{+\infty}}{\R}$ be
\begin{dgroup*}
\begin{dmath*}
f(t) = n \log t - t 
\end{dmath*},
\begin{dmath*}
   g(t) = \exp{f(t)} 
\end{dmath*}.
\end{dgroup*}
$g$ is integrand function.
Their   derivatives are respectively
\begin{dgroup*}
\begin{dmath*}
   \D{f(t)}{t} = \frac{n}{t} - 1
\end{dmath*},
\begin{dmath*}
   \D[2]{f(t)}{t} = -\frac{n}{t^{2}} 
\end{dmath*},
\begin{dsuspend}
   and
\end{dsuspend}
\begin{dmath*}
   \D{g(t)}{t} = g(t) \D{f(t)}{t}
\end{dmath*},
\begin{dmath*}
   \D[2]{f(t)}{t} = 
   \left[ \D[2]{f(t)}{t} + \left( \D{f(t)}{t} \right)^{2} \right] g(t) 
\end{dmath*}.
\end{dgroup*}
Since $g$ never vanishes,
$\D{g(t)}{t}=0$  if and only if
$\D{f(t)}{t}$ , \ie, if and only if $t=n$.


Since $\D[2]{f(t)}{t} = -\frac{1}{n}<0$ for every $t\in\interval[open
right]{0}{+\infty}$ and 
\begin{dmath*}
   \left( \D[2]{g(t)}{t} \right)_{\mathrlap{t=n}} = \left( \D[2]{f(t)}{t}
   \right)_{\mathrlap{t=n}}
   g(n) 
\end{dmath*},
the integrand function 
 has a maximum at $t=n$ with value $g(n) = n^{n}
\exp{-n}$.
$\D[2]{g(t)}{t}=0$ if and only if 
\begin{dmath*}
- \frac{n}{t^{2}} + \left(  \frac{n}{t} - 1 \right)^{2} = 0 
\end{dmath*},
\ie,
\begin{dmath*}
t^{2} -2n t + n^{2} - n = 0 
\end{dmath*}.
The two solutions of this equation  are $ t_{1,2} = n \pm \sqrt{n}$.

The saddle point recipe is:
approximate the integrand by Taylor expanding 
$f$ around  its maximum (which is also a maximum
of the integrand function itself).
Notice that we are \emph{not}  using Taylor expansion of the integrand itself, this is
because the latter  would
result  in a rather poor approximation if only lower-order terms had to be 
kept (\eg, second-order Taylor expansion of the integrand itself means that the
integrand is approximated with a parabola).

\begin{approfondimento}
Here, we are dealing with integrals with respect to a \emph{real}
variable. The saddle point method applies more generally to contour integrals in
 a complex domain. In the latter case, the kind of calculations are more or less
the same, however there is one big difference: additional work is needed which
consists of the deformation of the contour of integration. These topics are
refreshed in \cref{app:saddle}.
\end{approfondimento}

By Taylor expanding $f(t)$ around $t=n$ we have
\begin{dmath*}
   f(t) \approx f(n) + \left( \D{f(t)}{t} \right)_{t=n} (t-n) + \frac{1}{2}
   \left( \D[2]{f(t)}{t} \right)_{t=n} (t-n)^{2} 
\end{dmath*},
and since 
\begin{dgroup*}
   \begin{dmath*}
f(n) = n \log n - n 
\end{dmath*}
   \begin{dmath*}
      \left( \D{f(t)}{t} \right)_{t=n}  = 0 
   \end{dmath*},
   \begin{dmath*}[compact]
      \left( \D[2]{f(t)}{t} \right)_{t=n}  = - \frac{n}{n^{2}} = \frac{1}{n}
\end{dmath*}
\end{dgroup*}
we get
\begin{dmath*}
f(t) \approx \left( n \log n - n  \right) -\frac{1}{2n}  (t-n)^{2}  
\end{dmath*}.
Thus
\begin{dmath*}
   \GammaFunc{n+1} = 
   \Integrate{ \exp{f(t)}}{t,0,+\infty}
\approx 
\Integrate{ \exp{f(n)} \exp{ \frac{1}{2} \left( \D[2]{f(t)}{t} \right)_{t=n}
      \left( t-n\right)^{2}}}{t,0,+\infty}
= 
n^{n} \exp{-n} \Integrate{ \exp{\frac{1}{2n} (t-n)^{2}}}{t,0,+\infty}
\end{dmath*}.
For large $n$, we make a very small error by extending the integration to the whole real
axis, graphically this means that in \cref{fig:texp} the area under the Gaussian tail in the negative half
plane becomes negligible. The integral becomes a Gaussian integral over the
whole real line  which is evaluted exactly in closed form:
\begin{dmath*}[compact]
\Integrate{\exp{\frac{1}{2n} (t-n)^{2}}}{t,0,+\infty}
\approx 
\Integrate{\exp{\frac{1}{2n} (t-n)^{2}}}{t,\R} = 
\sqrt{2 \pi n} 
\end{dmath*}.
\begin{exercise}
Show that
\begin{dmath*}
\int_{0}^{+\infty} 
\Integrate{\exp{\frac{1}{2n} (t-n)^{2}}}{t,0,+\infty}  = \sqrt{\frac{n
      \pi}{2}} \left( 1 + \Erf{\sqrt{\frac{n}{2}}} \right) 
\end{dmath*}
where the error function has been used.  Give a bound to the error in the approximation
\begin{dmath*}
\Integrate{\exp{\frac{1}{2n} (t-n)^{2}}}{t,0,+\infty}
\approx 
\Integrate{\exp{\frac{1}{2n} (t-n)^{2}}}{t,\R}
\sqrt{2 \pi n} 
\end{dmath*}
\begin{hint} 
   The ratio of the two results is $\frac{1}{2} \left( 1 + \Erf{\sqrt{\frac{n}{2}}}
\right)$. Plot it. For ``large'' $n$ the ratio is $\approx 1$.
\end{hint}
\end{exercise}

\section{Stirling series}

Stirling's formula gives indeed the leading term of the following asymptotic
series, now called the Stirling series:


This is not a convergent series.
However, it is an asymptotic series, in fact there are a number of terms thet
improve accuracy, after which however accuracy gets worse. This is demonstarted
in

\section{A modern elementary proof\label{sec:Michael}}
Only 
the basic of calculus are required 
in order to prove two  preliminary  disequalities. No additional machinary is
needed%
\autocite[The original paper is][]{Michael:2002}.

%Once again, the starting point is
%\begin{displaymath}
%n! = \Gamma(n+1) = \int_{0}^{+\infty} t^{n} \E^{-t} \udiff{t} \eqspace .
%\end{displaymath}
%By making the substitution 
%\begin{displaymath}
%t = \xi \sqrt{n} + n \eqspace ,
%\end{displaymath}
%(the Jacobian of the transformation is $\sqrt{n}$) we get the modified
%representation
%\begin{eqnarray*}
%n! &=& \int_{-\sqrt{n}}^{+\infty} \left( \xi \sqrt{n} + n \right)^{n} \E^{-
%\left( \xi \sqrt{n} + n \right)} \sqrt{n} \udiff{\xi}  \\
%&=& n ^{n+\frac{1}{2}} \E^{-n} \int_{-\sqrt{n}}^{+\infty} \left( 1 +
%\frac{1}{\sqrt{n}} \xi \right)^{n} \E^{- \xi \sqrt{n}} \udiff{\xi} \eqspace .
%\end{eqnarray*}
%Now, we have a new goal, namely, to prove that 
%\begin{displaymath}
%\int_{-\sqrt{n}}^{+\infty}  \left( 1 + 	\frac{1}{\sqrt{n}}\xi  \right)^{n}
%\E^{-\xi \sqrt{n}} \udiff{\xi } \rightarrow \sqrt{2\pi}  \eqspace ,
%\end{displaymath}
%as $n\rightarrow +\infty$.
\appendix
\section{Gamma function warm up}
\label{sec:Gamma}
The basic properties of Euler's gamma function are refreshed%
\autocite[For completeness,
refer to, \eg, the classic book of][]{Magnus.Oberhettinger.ea:1953}.

There is a number of equivalent ways to define the gamma function%
\autocite[Our exposition follows mostly][\S~4.1]{Freitag.Busam:2005}.
Here we will discuss:
\begin{itemize}
   \item Euler's integral representation together with analytic continuation.
   \item characterization via the Bohr-Mollerup theorem.
   \item characterization via Willard theorem.
\end{itemize}

The gamma function extends the notion of the factorial to real and complex
numbers. The exact way in which it ``extends'' the factorial is precisely the content of the
characterizations theorems by Bohr-Mollerup and Willard.
However, gamma function is extremely useful well beyond the factorials. 

\subsection{Integral representation}
\label{sec:gamma integral}

\begin{definition}[Gamma function]
   \label{def:Gamma}
The gamma function 
$\GammaFunc{z}$ of a complex number $z$ can be defined by the Euler's integral representation 
\begin{dmath}[label={Gamma}]
   \GammaFunc{z} = \Integrate{ t^{z-1} \exp{-t} }{t, 0, +\infty}
   %\condition{$\forall z\in\C$ such that $\Re z > 0$}
\end{dmath}
valid for every $z\in\C$ such that $\Re z > 0$,
\emph{together with its analytic
   continuation}.
\end{definition}

\begin{remark}
   The definition is \emph{not} \cref{eq:Gamma} alone. It is \cref{eq:Gamma}
   for $\Re{z} >0$ plus analytic continuation. Both points will be throughly
   discuss later.
\end{remark}

%\begin{remark}
Let us clarify better \cref{eq:Gamma}.
      By definition of the complex powers,
      $t^{z-1} \coloneqq \exp{ (z-1) \log{t} }$, where $\log{t}$ is the
      \emph{real-valued} logarithmic function of the real variable $t$, while
      the exponential is the complex-valued exponential function.
      Integration in \cref{eq:Gamma} is over the dummy variable $t$ running over the real interval
$\interval[open right]{0}{+\infty}$,
\ie, we are integrating with respect to a \emph{real} (and not a
complex) variable. 

   In order \cref{def:Gamma} makes sense, we need to prove that 
\begin{inparaenum}[(a)]
\item the integral in \cref{eq:Gamma} is convergent if and only if $\Re{z}>0$;
\item the integral in \cref{eq:Gamma} defines an \emph{holomorphic} function in the half plane 
   $\Re{z} >0$;
\item study its analytic continuation for $\Re z \leq 0$.
\end{inparaenum}
Let us discuss all these three points.

\begin{lemma}

   $\Integrate{ t^{z-1} \exp{-t}}{t, 0, +\infty}$ is convergent
   \emph{if and only if} $\Re{z}> 0$.
\end{lemma}
\begin{proof}
   Consider the following integrals:
   \begin{dgroup*}
      \begin{dmath}[label={Int1}]
	 \Integrate{ t^{z-1} \exp{-t}}{t,0,1}
      \end{dmath}
      \begin{dmath}[label={Int2}]
	 \Integrate{ t^{z-1} \exp{-t}}{t,1,+\infty}
      \end{dmath}
   \end{dgroup*}
   We have 
   \begin{dmath*}
      \abs{ t^{z-1} \exp{-t}} = t^{\Re{z}-1} \exp{-t}
   \end{dmath*}.
   For $t>0$, $\exp{-t} <1$ and 
   we have 
   \begin{dmath*}[compact]
      \Integrate{\abs*{t^{z-1} \exp{-t}}}{t,0,1}
      =
      \Integrate{ t^{\Re{z}-1} \exp{-t}}{t,0,1}
      \leq 
      \Integrate{ t^{\Re{z}-1} }{t,0,1}
   \end{dmath*},
   and the latter (real) integral is convergent if and only if $\Re{z} >0$.
\end{proof}

\begin{exercise}
   Show that for every $z\in\C$ such that $\Re{z}>0$, $\Gamma(z)$ can be
   equivalently written  as
\begin{dmath*}
   \Gamma(z) = 2 \Integrate{ \xi^{2z-1} \exp{-\xi^{2}} }{\xi, 0, +\infty}
\end{dmath*}
or
\begin{dmath*}
\Gamma(z) =  \Integrate{\log[z-1]{\frac{1}{\xi}} }{\xi,0,1} 
\end{dmath*}
\begin{hint}
Apply a suitable change of variables. Why is it possible to performe a change of
variables?
\end{hint}
\end{exercise}

\begin{lemma}
   $\Integrate{ t^{z-1} \exp{-t}}{t, 0, +\infty}$ defines an
   \emph{holomorphic} function of $z$ for every $z\in\C$ such that $\Re{z}>0$.
\end{lemma}

\begin{proof}

   There are two ways to show this:
   \begin{itemize}
      \item Direct calculation of 
	 $\D{}{z}\Integrate{ t^{z-1} \exp{-t}}{t, 0, +\infty}$;
      \item Application of the Morera's theorem.
   \end{itemize}

\end{proof}




It can be easily verified that by definition
\begin{dmath*}[compact]
   \GammaFunc{1} = \Integrate{ \exp{-t}}{t,0,+\infty} 
   = \left[ -\exp{-t} \right]_{t=0}^{t=+\infty} = 1
\end{dmath*}.
Moreover, by using the standard result of the Gaussian integral, namely%
\footnote{There are several ways to prove \cref{eq:gaussian}. Refer to, \eg,
   \cite{Iwasawa:2009,Boros.Moll:2004}.},
\begin{dmath}[compact,label={gaussian}]
\Integrate{ \exp{-\xi^{2}} }{t,0,+\infty} = \frac{\sqrt{\pi}}{2}
\end{dmath},
we readily obtain
\begin{dmath}[compact]
\GammaFunc{\frac{1}{2}} = 
\Integrate{ t^{-\frac{1}{2}} \exp{-t}}{t,0,+\infty}
=
\Integrate{ \frac{1}{\xi}  \exp{-\xi^{2}} 2\xi }{t,0,+\infty}
=
 2
 \underbrace{\Integrate{ \exp{-\xi^{2}}  }{t,0,+\infty}}_{\sqrt{\pi}/2}
=  \sqrt{\pi} 
\condition*{\xi= \sqrt{t}}
\end{dmath},
where the factor $2\xi$ in the second integral above comes from the Jacobian of
the transformation $\xi = \sqrt{t}$. Notice: the original integral is an
integral however the real line, we are allowed to change variables in this way,
the square root denotes the real square root.

We will see later that, by using \cref{eq:rec}, the gamma function can be
simplified for all half-integer arguments.

\subsection{Functional equation}
\begin{theorem}[recursive relation]
   \label{thm:rec}
   For every $z\in \C$, $\Re{z}>0$, 
\begin{dmath}[label={rec},frame]
   \GammaFunc{z+1} = z \GammaFunc{z} 
\end{dmath}.
\end{theorem}
\begin{proof}
   Integrating by parts in \cref{eq:Gamma} yields
\begin{dmath*} 
   \GammaFunc{z} = \underbrace{\left. \frac{t^{z}}{z} \exp{-t}
\right|_{0}^{+\infty}}_{0}  +
\frac{1}{z} 
\underbrace{ \Integrate{t^{z} \exp{-t}}{t,0,+\infty}}_{\GammaFunc{z+1} } 
\end{dmath*}.
(The first term in the right-hand side vanishes as shown in \cref{ex:rec}.)
So \cref{eq:rec} holds.
\end{proof}
\begin{exercise}
   \label{ex:rec}
   In connection with the proof of \cref{thm:rec}, prove that 
\begin{dmath*}
\left. \frac{t^{z} \exp{-t}}{z} \right|_{0}^{+\infty}  =
0 
\end{dmath*},
in particular 
\begin{dmath*}
\lim_{t\rightarrow +\infty} t^{z} \exp{-t} = 0  
\end{dmath*}
for every $z\in\C$. 
\begin{hint}
By definition, $t^{z} \exp{-t} \coloneqq \exp{z \log t - t}$. Consider
\begin{dmath*}[compact]
\lim_{t\rightarrow+\infty} \abs*{t^{z} \exp{-t}} = 
\lim_{t\rightarrow+\infty} \abs*{ \exp{ z \log t - t}} =
\lim_{t\rightarrow+\infty} \abs{ \Re z \log t -t } 
\end{dmath*};
show that this limit is zero for every $z\in\C$. This implies also
$\lim_{t\rightarrow+\infty} t^{z} \exp{-t} =0 $ (why?).
\end{hint}
\end{exercise}


\Cref{eq:rec} has far-reaching conseguences.
First of all, \cref{eq:rec} implies that 
\begin{dmath}[label={gamman+1},frame]
   \GammaFunc{n+1} = \Factorial{n} 
\end{dmath}
for every $n\in\N$.
Let us prove it by induction.
First of all, 
\begin{dmath*}[compact]
   \GammaFunc{2} = \GammaFunc{1+1} = 1 \GammaFunc{1} = 1 
\end{dmath*},
which is equal to $\Factorial{1} =1 $.
Moreover,  assuming $\GammaFunc{n+1} = \Factorial{n}$ let us prove that 
$\GammaFunc{n+2} = \Factorial{(n+1)}$:
\begin{dmath*}[compact]
   \GammaFunc{(n+1) + 1} = (n+1) \GammaFunc{n+1}  = (n+1) \Factorial{n} =
   \Factorial{n+1} 
\end{dmath*}.
This completes the proof. Notice also that \cref{eq:gamman+1} is consistent
with the position $\Factorial{0}=1$, in fact
\begin{dmath*}[compact]
   \GammaFunc{1} = 1 = \Factorial{0} 
\end{dmath*}.

As a consequence, the gamma function can be used to extend the notion of
factorials to the \emph{complex} 
numbers: $z! \coloneqq\Gamma(z+1)$. 

An even more important conseguence of \cref{eq:rec} is that it can be used
to implement the \emph{analytic continuation} of $\GammaFunc{z}$ outside the original domain of definition
of the integral representation \cref{eq:gamma}.
Let us now discuss better this point.

We know that $\Gamma{z}$ in \cref{eq:gamma} is 
\begin{inparaenum}[(a)]
\item analytic for $\Re{z}>0$;
\item satifies \cref{eq:rec} for $\Re{z}>0$.
\end{inparaenum}
Consider 
\begin{dmath*}
   \GammaFunc{z} = \frac{\Gamma{z+1}}{z} 
   \condition*{\Re{z}>0}
\end{dmath*}.
$\Gamma{z+1}$ on the right-hand side is indeed defined for $\Re (z+1)
>0$, \ie, $\Re z >-1$.
$\GammaFunc{z+1} /z $ thus coincides with the original definition of
$\GammaFunc{z}$ 
for $\Re z>0$,  but it is also valid for $-1<\Re z < 0$, allowing the
analytic continuation 
of $\Gamma{z}$ to the strip $-1<\Re z <0$.

For example, let us 
compute $\GammaFunc{-1/2}$.
We have
\begin{dmath*}[compact]
   \GammaFunc{-\frac{1}{2} )} = \frac{\GammaFunc{- \frac{1}{2} +
	 1}}{-\frac{1}{2}} = -2 \GammaFunc{\frac{1}{2} } = -2 \sqrt{\pi}
\end{dmath*}.

Repeating the same reasoning, we can further extend the definition  of
$\GammaFunc{z}$ via analytic continuation to all other points of the half-plane $\Re
z <0$.
In this way, 
\begin{dmath}[label={gamman},frame]
   \GammaFunc{z} = \frac{\GammaFunc{z+n}}{\Pochhammer{z}{n}}
   \condition*{\forall n\in\N}
\end{dmath},
where $\Pochhammer{z}{n}$
is the Pockhammer symbol of index $n$.
\begin{definition}[Pochhammer's symbol]
   For every $z\in\C$ and $n\in\N$, the Pochhammer's symbol
   $\Pochhammer{z}{n}$ is defined by induction on $n$:
   \begin{itemize}
      \item $\Pochhammer{z}{1}= z$ and 
      \item $\Pochhammer{z}{n+1} = (z+n) \Pochhammer{z}{n}$  for every $n\geq
	 1$.
   \end{itemize}
  Sometimes it is useful to define also $\Pochhammer{z}{0} =1$.
\end{definition}
In other words,
\begin{dmath*}
   \Pochhammer{z}{n}= z (z+1) (z+2) \cdots (z+n-1) 
\end{dmath*}.
Notice that $\Pochhammer{1}{1} = 1$ and $\Pochhammer{1}{n+1} = (n+1)
\Pochhammer{1}{n}$, that is  $\Pochhammer{1}{n} = \Factorial{n}$.

\begin{lemma}
$\GammaFunc{z}$ has simple poles at $z=-n$, with
$n\in\N$, with residues given by 
\begin{dmath}
   \Res{z=-n}{\Gamma(z)} = \lim_{z\rightarrow -n} (z+n) \Gamma(z) =
\lim_{z\rightarrow -n} (z+n) \frac{\Gamma(z+n+1)}{(z)_{n+1}} =
\lim_{z\rightarrow -n} \frac{\Gamma(z+n+1)}{(z)_{n}} = 
\frac{(-1)^{n}}{n!} 
\end{dmath}.
\end{lemma}
\begin{exercise}
   Prove that $\Pochhammer{-n}{n}= (-1)^{n} \Factorial{n}$.
\end{exercise}

The plot of $\GammaFunc{z}$ for real values of $z$ is given in
\cref{fig:gamma_plot}.

\begin{figure}
   \centering
\asyinclude[inline=true]{./Asymptote/gamma.asy}
\caption{Plot of gamma function for real values of its argument}
\end{figure}


\subsection{Double factorial notation}
One often encounters products of the odd positive integers and products of the
even positive intergers.
For convenience, one introduces the following double factorial notation.
\begin{definition}[Double factorials]
By
induction,  define
$\DblFactorial{0} = \DblFactorial{1} = 1$ and for 
every $n\in\N$, $n\geq 1$,
\begin{dgroup*}
\begin{dmath}[label={dblfactorial2n+1}]
   \DblFactorial{(2n+1)} = (2n+1) \DblFactorial{(2n-1)} 
\end{dmath}
\begin{dsuspend}
   and
\end{dsuspend}
\begin{dmath}[label={dblfactorial2n}]
   \DblFactorial{(2n)} = (2n) \DblFactorial{(2n-2)} 
\end{dmath}.
\end{dgroup*}
\end{definition}
In other words,
\begin{dgroup*}
   \begin{dmath*}
      \DblFactorial{(2n+1)} = 1 \times 3 \times 5 \times \cdots \times (2n+1) 
   \end{dmath*},
   \begin{dmath*}
      \DblFactorial{(2n)} = 2 \times 4 \times 6 \times \cdots \times (2n) 
   \end{dmath*}.
\end{dgroup*}
Furthermore, it is useful (see, \eg, the next exercise) to define
$\DblFactorial{(-1)} =1$.
\begin{lemma}
\begin{dgroup*}
   \begin{dmath}[label={dblfactorial2nclose}]
      \DblFactorial{(2n)} = 2^{n} \Factorial{n} 
   \end{dmath},
   \begin{dmath}[label={dblfactorial2n+1close}]
      \DblFactorial{ (2n+1)} = \frac{\Factorial{(2n+1)}}{2^{n} \Factorial{n}}
   \end{dmath}.
\end{dgroup*}
\end{lemma}
\begin{proof}
   Heuristic argument to prove \cref{eq:dblfactorial2nclose}:
\begin{dmath*}
   \DblFactorial{2n}= 2 \times 4 \times 6 \times \cdots \times 2n 
= (2 \times 1) \times (2\times 2) \times (2\times 3) \times \cdots \times
(2\times n) 
= 2^{n} \Factorial{n} 
\end{dmath*}.
Rigorous proof of \cref{eq:dblfactorial2nclose} is by induction, as follows.
Notice however that the rigorous proof by induction gives no help in deriving
the formula in \cref{eq:dblfactorial2nclose},
it simply gives a rigorous proof of it once the formula has been already found.
Proof by induction of \cref{eq:dblfactorial2nclose}:
$\DblFactorial{0}  = 2^{0} \Factorial{0} = 1$ (so the formula holds for $n=0$) and
\begin{dmath*}[compact]
   \DblFactorial{(2n+2)} = (2n+2) \DblFactorial{(2n)} = 
   (2n+2) 2^{n} \Factorial{n} = 2 (n+1) 2^{n} \Factorial{n} = 2^{n+1}
   \Factorial{(n+1)}
\end{dmath*}.
\Cref{eq:dblfactorial2n+1close} follows  from
\begin{dmath*}
   \DblFactorial{(2n+1)} = \frac{\Factorial{(2n+1)}} {\DblFactorial{(2n)}} 
\end{dmath*}.
(Again, rigorous proof is by induction.)
\end{proof}

\begin{lemma}
   Show that 
   \begin{dmath*}[compact]
   \GammaFunc{\frac{1}{2} +n}
   = \frac{\DblFactorial{(2n-1)}}{2^{n}} \GammaFunc{\frac{1}{2}} 
   = \frac{\DblFactorial{(2n-1)}}{2^{n}} \sqrt{\pi} 
   \condition*{\forall n \in \N}
\end{dmath*}.
\end{lemma}
\begin{proof}
Let us check the formula for some $n$
\begin{dgroup*}
   \begin{dmath*}[compact]
   \GammaFunc{\frac{1}{2} +  1} =  \frac{1}{2} \GammaFunc{\frac{1}{2}}
\end{dmath*},
   \begin{dmath*}[compact]
\Gamma\left(\frac{1}{2} +  2\right) =  \left( \frac{1}{2} + 1\right)
\GammaFunc{\frac{1}{2} +  2} = \left( \frac{1}{2} +1\right)
   \GammaFunc{\frac{1}{2}+1}
   = \frac{3}{2} \frac{1}{2} \GammaFunc{\frac{1}{2}  }  
\end{dmath*},
\end{dgroup*}
etc.
Rigorous proof is by induction: the formula gives the correct result for the
case $\GammaFunc{\frac{1}{2} + 1}$ (indeed, even for the case $n=0$ assuming 
$\DblFactorial{(-1)} =1$ is understood); for $n\geq 1$,
\begin{dmath*}
   \GammaFunc{\frac{1}{2} + n + 1 } = 
   \left( \frac{1}{2} + n \right) \GammaFunc{\frac{1}{2} + n}= 
   \frac{1 + 2n}{2} \frac{\DblFactorial{(2n-1)}}{2^{n}} \GammaFunc{\frac{1}{2}
   }  = 
   \frac{\DblFactorial{(2n+1)}}{2^{n+1}} \GammaFunc{\frac{1}{2}}
\end{dmath*},
this completes the proof.
\end{proof}
\subsection{Another definition of $\Gamma(z)$ and a proof of Wallis formula}
Another equivalent way to define $\Gamma(z)$ is through the formula
\begin{dmath}[label={gammalim}]
   \GammaFunc{z} = \lim_{n\rightarrow+\infty} \frac{\Factorial{n}}
   {\Pochhammer{z}{n+1}} n^{z} 
\end{dmath}.


Essentially, the argument goes as follows%
\autocite[See, \eg, ][\S~1.1]{Magnus.Oberhettinger.ea:1953}.
 For $n\in\N$ and $\Re z>0$, repeated integration by parts yields
\begin{dmath*}
   \Integrate{
      \left( 1 - \frac{t}{n} \right)^{n} t^{z-1} }{t,0,+\infty}
  = 
  \frac{\Factorial{n}}{\Pochhammer{z}{n+1}}n^{z}
\end{dmath*}.
(Check as an exercise.)
In the limit $n\rightarrow+\infty$ the left-hand side above becomes 
$\Integrate{t^{z-1} \exp{-t}}{t,0,+\infty}$.
However, care
is needed in taking this limit.

We can use the equivalent definition above of $\GammaFunc{z}$ to give a short proof of Wallis formula.
Since we know that 
\begin{dmath*}[compact]
   \GammaFunc{-\frac{1}{2} } = -2 \GammaFunc{ \frac{1}{2} } = -2\sqrt{\pi}
\end{dmath*},
we can write
\begin{dmath*}
   \GammaFunc{- \frac{1}{2} } = \lim_{n\rightarrow+\infty} \frac{\Factorial{n}}{
      \Pochhammer{\frac{1}{2}}{n+1} } n^{-\frac{1}{2}}  
\end{dmath*}
We need to compute explicitly the Pochhammer symbol in this formula:
\begin{dmath*}
   \Pochhammer{-\frac{1}{2}}{n+1} = 
   -\frac{1}{2} \Pochhammer{\frac{1}{2}}{n}
\end{dmath*}
and using the double factorial notation we have
\begin{dmath*}[compact]
\Pochhammer{\frac{1}{2}}{n} = 
\frac{\DblFactorial{(2n-1)}}{2^{n}}  = \frac{\DblFactorial{(2n+1)}}{2^{n}
   (2n+1)} = \frac{\Factorial{(2n+1)}}{2^{2n}\Factorial{n} (2n+1)} = 
\frac{\Factorial{(2n)}}{2^{2n} \Factorial{n}} 
\end{dmath*}
Thus, 
\begin{dmath*}
   \GammaFunc{-\frac{1}{2}} = 
   \lim_{n\rightarrow+\infty}
   \frac{\Factorial{n}}{\Pochhammer{-\frac{1}{2}}{n+1}}
n^{-\frac{1}{2}} = \lim_{n\rightarrow+\infty}
\frac{2^{2n} (\Factorial{n})^{2} }{-\frac{1}{2} \Factorial{(2n)}} n^{-\frac{1}{2}} 
\end{dmath*},
from which it follows
\begin{dmath*}
   \sqrt{\pi} = \lim_{n\rightarrow+\infty} \frac{2^{2n}
      (\Factorial{n})^{2}}{\Factorial{(2n)}} n
^{-\frac{1}{2}} 
\end{dmath*},
which is the 
Wallis formula.

\subsection{Beta function}
\label{sec:Beta}

The beta function is defined by the integral representation
\begin{dmath}[label={beta},frame]
   \Beta{p, q} = \Integrate{t^{p-1} (1-t)^{q-1}}{t,0,1}
\end{dmath}
for every pair of complex numbers $(p,q)\in\C^{2}$, such that 
$\Re p>0$ and $\Re q >0$, and its analytic continuation.
\begin{exercise}
Show that the integral representation in \cref{eq:beta} converges for $\Re p
>0$ and $\Re q >0$.
\end{exercise}

First of all, notice that 
\begin{dmath}
   \Beta{q,p}=  \Beta{p,q}
   \condition{$\Re{p} > 0$ and $\Re{q} >0$}
\end{dmath}.
In fact,
\begin{dmath*}
   \Beta{q,p} = 
\Integrate{ t^{q-1} ( 1-t)^{p-1}}{t,0,1}
= -
\Integrate{ (1-\xi)^{q-1} \xi ^{p-1}}{\xi,1,0}
=
\Integrate{ (1-\xi)^{q-1} \xi ^{p-1}}{\xi,0,1}
= \Beta{p,q}
\condition*{\xi= -t}
\end{dmath*}.

Now, we prove an important formula.
\begin{theorem}
   \begin{dmath}[label={betagamma},frame]
      \Beta{u, v}= \frac{\GammaFunc{u} \GammaFunc{v}}{\GammaFunc{u+v}} 
\end{dmath}
for all $(u,v)\in\C^{2}$, $\Re u >0$, $\Re v >0$.
\end{theorem}
\begin{proof}
Step 1: consider
\begin{dmath*}
   \GammaFunc{u} \GammaFunc{v} = 
   \left(\Integrate{ t^{u-1} \exp{-t}}{t,0,+\infty}\right)
   \left(\Integrate{ t^{v-1} \exp{-t}}{t,0,+\infty}\right)
\end{dmath*}.
Step 2: replace $t = x^{2} $ and $s = y^{2}$, giving
\begin{dmath*}
   \GammaFunc{u} \GammaFunc{v} = 4
   \left(\Integrate{ x^{2u-1} \exp{-x^{2}}}{t,0,+\infty}\right)
   \left(\Integrate{ y^{2v-1} \exp{-y^{2}}}{t,0,+\infty}\right)
\end{dmath*}.
Step 3: Fubini's theorem applies allowing to convert the product of two integrals into one
double integral in the plane
\begin{dmath*}
   \GammaFunc{u} \GammaFunc{v} = 4
   \Integrate{ 
      \Integrate{ x^{2u-1} y^{2v-1} \exp{-\left( x^{2} + y^{2}
	    \right)}}{x,0,+\infty}
   }
   {y,0,+\infty}
\end{dmath*}
%\int_{0}^{+\infty} \udiff{x} \int_{0}^{+\infty}
%\udiff{y} x^{2u-1} y^{2v-1} \E^{-\left( x^{2}+y^{2}\right) }
Step 4: we change variables to polar coordinates in the plane:
\begin{dmath*}
   \GammaFunc{u} \GammaFunc{v} = 4 
   \Integrate{ \Integrate{ \rho^{2u+2v-1} \exp{-\rho^{2}} \cos[2u-1]{\vartheta}
      \sin[2v-1]{\vartheta}}{\vartheta, 0, \frac{\pi}{2}}}{\rho, 0,+\infty}
\end{dmath*}.
Step 5: Fubini's theorem allows to factorize the double integral in polar
coordinates into the product of two integrals:
\begin{dmath*}
   \GammaFunc{u} \GammaFunc{v} = 4 
%\Gamma(u) \Gamma(v) = 4 \left( \int_{0}^{+\infty} \rho^{2u +2v - 1}
%\E^{-\rho^{2}} \udiff{\rho} \right) \left( \int_{0}^{\frac{\pi}{2}}
%\cos^{2u-1} \vartheta \sin^{2v-1} \udiff{\vartheta}  \right)
%\end{displaeqnarrayymath}
\left(\Integrate{ \rho^{2u+2v-1} \exp{-\rho^{2}} }{\rho, 0, +\infty}\right)
\left(\Integrate{ \cos[2u-1]{\vartheta}
      \sin[2v-1]{\vartheta}}{\vartheta,0,\frac{\pi}{2}} \right)
\end{dmath*}.
Step 6:
integration over $\rho$ returns
\begin{dmath*}[compact]
\Integrate{ \rho^{2u+2v-1} \exp{-\rho^{2}} }{\rho, 0, +\infty} = 
\frac{1}{2}
\Integrate{ \left( \xi^{\frac{1}{2}} \right)^{2u+2v-1} \exp{-\xi}
   \xi^{-\frac{1}{2}}  }{\xi, 0, +\infty}
= 
\frac{1}{2}
\Integrate{\xi^{u+v-1} \exp{-\xi}}{\xi, 0, +\infty} 
= \frac{1}{2} \GammaFunc{u+v} 
     \condition*{\xi = \rho^{2}}
\end{dmath*}.
Step 7: integration over $\vartheta$ can be performed by changing variables: $t
= \cos[2]{\vartheta}$, thus $\sin[2]{\vartheta} = 1 -t $, the Jacobian is 
\begin{dmath*}
   \D{\arccos{\sqrt{t}}}{t} 
= - \frac{1}{2} (1 -t )^{-\frac{1}{2}} t^{-\frac{1}{2}}
\end{dmath*}
so
\begin{dmath*}[compact]
\Integrate{ \cos[2u-1]{\vartheta}
   \sin[2v-1]{\vartheta}}{\vartheta,0,\frac{\pi}{2}}= 
-\frac{1}{2} \Integrate{ t^{u-\frac{1}{2}}
   (1-t)^{v-\frac{1}{2}} (1-t)^{-\frac{1}{2}} t^{-\frac{1}{2}} }{t,1,0}
= \frac{1}{2} \Integrate{t ^{u-1} (1-t)^{v-1} }{t,0,1} = \Beta{u,v}
\end{dmath*}.
Thus
\begin{dmath*}[compact]
   \GammaFunc{u} \GammaFunc{v} = 4 \frac{1}{2} \GammaFunc{u+v} \frac{1}{2}
   \Beta{u,v} =
   \GammaFunc{u+v} \Beta{u,v} 
\end{dmath*}.
This completes the proof.
\end{proof}

We will use the beta function to compute quickly $I_{n}$ in \cref{eq:In} in
\cref{sec:Wallis}.
Show that 
\begin{dmath}[compact,frame]
   I_{n} =  
   \Integrate{\sin[n]{\vartheta}}{\vartheta,0, \frac{\pi}{2}} 
   =  \frac{1}{2}
   \sqrt{\pi} \frac{ \GammaFunc{\frac{n+1}{2}} }{\GammaFunc{\frac{n}{2}
	 +1 }}
\end{dmath}.
We have
\begin{dmath*}
   \Integrate{\sin[n]{\vartheta}}{\vartheta,0, \frac{\pi}{2}}  = 
   \Integrate{\xi^{n} \left( 1 - \xi^{2}\right)^{-\frac{1}{2}}}{\xi, 0, 1}
= \frac{1}{2} \Integrate{t^{\frac{n}{2}} \left( 1 - t\right)^{-\frac{1}{2}}
   t^{-\frac{1}{2}}}{t,0,1} 
= \frac{1}{2} \Beta {\frac{n-1}{2},\frac{1}{2}}
\condition{where $\xi = \sin{\vartheta}$ and $t=\xi^{2}$}
\end{dmath*}.
Using \cref{eq:betagamma} we have
\begin{dmath*}[compact]
   \Integrate{\sin[n]{\vartheta}}{\vartheta,0, \frac{\pi}{2}}  = 
   \frac{1}{2} \frac{\GammaFunc{\frac{n+1}{2}} \GammaFunc{\frac{1}{2}}}
   {\GammaFunc{\frac{n+1}{2} + \frac{1}{2} }} = \frac{1}{2}
\sqrt{\pi} \frac{ \GammaFunc{ \frac{n+1}{2} }}{ \GammaFunc{  \frac{n}{2}
+1 }} 
\end{dmath*}.
For $n=2k$,  the formula simplifies to
\begin{dmath*}
   I_{2k} = \frac{\sqrt{\pi}}{2} \frac{\GammaFunc{k + \frac{1}{2}}}
   {\GammaFunc{k+1}} = \frac{\sqrt{\pi}}{2} \frac{ \DblFactorial{(2k-1)}}{2^{k}
      \Factorial{k} }
   \sqrt{\pi}  = \frac{\pi}{2} \frac{\DblFactorial{(2k-1)}}{\DblFactorial{(2k)}}
\end{dmath*},
while for $n=2k+1$ the formula returns
\begin{dmath*}
   I_{2k+1} = \frac{\sqrt{\pi}}{2} \frac{ \GammaFunc{k+1} }{ \GammaFunc{k+1 +
	 \frac{1}{2}}} = \frac{\sqrt{\pi}}{2} \frac{ \Factorial{k} 2^{k+1}}
   {\DblFactorial{(2k+1)}}
   \frac{1}{\sqrt{\pi}} = \frac{\DblFactorial{(2k)}}{\DblFactorial{(2k+1)}} 
\end{dmath*},
as per the results of \cref{sec:Wallis}.
\subsection{Reflection formula}
In this section we prove the important \emph{reflection} formula using residue
theory. There are also other ways to prove it, \eg, Weierstrass infinite-product
definizion of $\GammaFunc{z}$ leads directy to this result.
\begin{theorem}[reflection formula]
   \begin{dmath}[label={ref},frame]
      \GammaFunc{z} \GammaFunc{1-z} = \frac{\pi}{\sin{\pi z} }  
\end{dmath}.
\end{theorem}
\endinput
\begin{proof}
We begin with $0< \Re z < 1 $, for which the integral representations are valid,
then we apply analytic continuation.
\par
For Eq.~\eqref{eq:betagamma}, 
\begin{equation}\label{eq:G(z)G(1-z)}
\Gamma(z ) \Gamma(1-z) = \Gamma( z +1 -z) \beta(z, 1-z) = \beta(z, 1-z) =
\int_{0}^{1} t^{z-1} (1-t)^{-z} \udiff{t} \eqspace , 
\end{equation}
for $0<\Re z <$.
The integral can be computed exactly by  contour integration in the complex
plane.
\par
Let $f(w )$ be the complex-valued function of the complex variable
$w\in\C$ defined by
\begin{displaymath}
f(w) = w^{z-1} (w-1)^{-z} \eqspace .
\end{displaymath}
\begin{remark}
$f(w)$ is \emph{not} $w^{z-1} (1-w)^{-z}$.
\end{remark}
We choose the branch cuts
\begin{displaymath}
0 \leq \arg w < 2\pi \eqspace, \qquad 0 \leq \arg(w-1) < 2\pi \eqspace ,
\end{displaymath}
Consider $\oint_{\gamma} f(w) \udiff{w} $ where $\gamma$ is the contour
consisting in the straight joining the points $w=0$ and $w=1$, with
indentations at $w=0$ and $w=1$.
\par
If the raddi of indentation are made to approach  zero,  the contributions from
the indentation vanish in that limit, since
\begin{eqnarray*}
\lim_{w\rightarrow 0} w f(w) &=& \lim_{w\rightarrow 0 } w^{z} (w-1)^{-z} = 0
\eqspace ,\\
\lim_{w\rightarrow 1 } (w-1) f(w) &=& \lim_{w\rightarrow 0 } w^{z-1}
(w-1)^{-z+1} = 0 \eqspace .
\end{eqnarray*}
The residue at infinity is one,  and we get
\begin{displaymath}
\int_{0}^{1} t^{z-1} (1-t)^{-z} \left( \E^{-i\pi z} - \E^{i\pi z } \right) \udiff{t} =
-2\pi i  \eqspace ,
\end{displaymath}
hence
\begin{displaymath}
\int_{0}^{1} t^{z-1} (1-t)^{-z} \udiff{t} = \frac{-2\pi i }{\E^{-i\pi  z} -
\E^{i\pi z }} = \frac{\pi}{\sin \pi z } \eqspace .
\end{displaymath}
Pluggin into  Eq.~\eqref{eq:G(z)G(1-z)} completes the proof for $0<\Re z < 1$.
Then, apply analytic continuation.
\end{proof}
Setting $z= \frac{1}{2}$ in Eq.~\eqref{eq:G(z)G(1-z)} we obtain
\begin{displaymath}
\left[\Gamma\left( \frac{1}{2} \right)\right]^{2} = \frac{\pi}{\sin
\frac{\pi}{2}} = \pi \eqspace ,
\end{displaymath}
and taking the positive square root yields $\Gamma\left( \frac{1}{2}\right) =
\sqrt{\pi}$, in agreement with the previous result.
\par
Eq.~\eqref{eq:ref} shows immediately that $\Gamma(z)$ has no zeros, since
$\frac{\pi}{\sin \pi z }$ is never equal to zero.


\section{Steepest descent method\label{app:saddle}}
The 
stepeest descent method (also referred to as saddle point method) is a powerful tool to handle the problem of
evaluating  the
  asymptotic behavior (or at least its leading term) of parametric integrals of a certain form.
For example, it might be useful in evaluating the asymptotic behavior of functions
 for which a suitable integral representation is available
 \autocites[An introductory discussion of these topics is given, \eg,
 by][\S~11]{King.Billingham.ea:2003}[or][\S~6.]{Ablowitz.Fokas:2003}[See
 also][\S~1]{Zinn-Justin:2002}.

Let us begin with a special case of real integrals over one real variable (this
is known also as Laplace's method) and then generalize to contour integrals on
a  complex domain. Generalization to an arbitrary number of variables is not
covered here, even if such generalizations are often needed in the toolkit of
every physicists%
\autocite[Stepeest  descent method  generalizes also to
functional
integrations and path integrals, this is very important in dealing with
non-perturbative corrections in quantum mechanics and quantum field
theory. Refer to, \eg, ][]{Zinn-Justin:2002}.
%we restrict to integrals with respect to a
%\emph{real} variable 
%(Laplace method). This is enough for example to develop the Stirling's
%approximation of $\Gamma(x)$ for real values of $x$ as $x\rightarrow+\infty$ and thus deriving the
%Stirling formula of $n!$.
%Then we will move to the more general case involving contour integrals in the
%complex
%plane. This is the case if you seek for example the asymptotic behavior of  
%$\Gamma(z)$
%for complex values of $z$ as $\modul{z} \rightarrow +\infty$.
%\par
\subsection{Real case (alias, Laplace method)}
\subsection{Complex case}

\printbibliography
\end{document}
